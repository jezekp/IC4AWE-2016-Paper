\documentclass[a4paper,twoside]{article}

\usepackage{epsfig}
\usepackage{subfigure}
\usepackage{calc}
\usepackage{amssymb}
\usepackage{amstext}
\usepackage{amsmath}
\usepackage{amsthm}
\usepackage{multicol}
\usepackage{pslatex}
\usepackage{apalike}
\usepackage{SCITEPRESS}     % Please add other packages that you may need BEFORE the SCITEPRESS.sty package.

\subfigtopskip=0pt
\subfigcapskip=0pt
\subfigbottomskip=0pt

\begin{document}

\title{MoBio - A mobile system for collecting electrophysiological data\subtitle{Preparation of Camera-Ready Contributions to SCITEPRESS Proceedings} }

\author{\authorname{Petr Je\v{z}ek\sup{1} and Roman Mou\v{c}ek\sup{1}}
\affiliation{\sup{1}New Technologies for the Information Society, 
              Department of Computer Science and Engineering,
              Faculty of Applied Sciences,
              University of  West Bohemia,
              Univerzitn\'{i} 8,
              306 14  Plze\v{n},
              Czech Republic}
\email{jezekp@ntis.zcu.cz, moucek@kiv.zcu.cz}
}   

\keywords{The paper must have at least one keyword. The text must be set to 9-point font size and without the use of bold or italic font style. For more than one keyword, please use a comma as a separator. Keywords must be titlecased.}

\abstract{The abstract should summarize the contents of the paper and should contain at least 70 and at most 200 words. The text must be set to 9-point font size.}

\onecolumn \maketitle \normalsize \vfill

\section{\uppercase{Introduction}}
\label{sec:introduction}

\noindent 
There are lot of factors affecting human health. Factors such as genetics, environmental influence or internal state of an individual cannot be easy measured. On the other hands there are factors as blood pressure, glucose level or heart rate that can be measured relatively easy, non-invasively by cheap sensors. For a long time electrophysiological experiments have been conducted in laboratories equipped by common desktop computers and non-transferable measuring devices. Fortunately situation is rapidly changing in these days. The one of reasons is increasing popularity of smart devices as phones or tablets. According to eMarketer \cite{emark} two billions of people will own a smart-phone in 2016. Simultaneously a lot of relatively cheap sensors for measurement of potentials from the human body is available on the market. These sensors often transfer data wirelessly, it means they can be easily read and processed by smart devices. The obtained data can be used by two fundamental intersecting ways. 

The first assistive technology approach serves to stimulating, maintaining or improving functional capabilities of people with special needs including disabled people or aging population. Getting independence and self-sufficiency increases quality of life in general. 

The second approach is focused on sportsmen or actively living people. Measurements are based on a monitoring of persons when they are performing specific activities (e. g. running or long distance walking). The data are used to display the current status and a long term monitoring of the fitness level. 

We operate a complete equipped laboratory \cite{10.3389/fninf.2014.00020} for electrophysiology measurements. We are focused  mainly on Electroencephalography (EEG) and Event-related potentials (ERP). Except of the fixed laboratory we also operate a so-called mobile laboratory equipped by a set of laptops and portable measuring devices for performing experiments out of the laboratory. With advancing efforts for extension the laboratory to collect broader spectrum of data (e. g. blood pressure, Elektrocardiogram (EKG), glucose level, heart rate) we are proposing here a prototype of mobile client for collecting data from wireless devices. The client provides API for connecting to limited set of devices available on the market and enables synchronization with the remote storage.

\section{\uppercase{state of the art}}
\label{sec:state-of-the-art}

\noindent
Neuroinformatics community identified problems with a long-term description, storage and management of experimental data/metadata \cite{CRCNS}. As a member of International Neuroinformatics Coordinating Facility (INCF) \cite{INCF} we are being developing a system for long term storage and management of EEG/ERP experiments - EEGBase \cite{ISI:000306821100004}. A mobile EEG-Base client \cite{10.3389/conf.fninf.2013.09.00046} is a supplementary Android tool that enables collecting experiments out of the laboratory and provides an on-line synchronization with EEGBase.
 
 According to \cite{Lowe2012242} appliactions using sensors can be divided into three categories. The first category Smart Phone Applications use either GPS or the
onboard kinematic sensors as the technologies of choice for monitoring exercise. The second category comprises of any system that uses a central controller
and an external sensor. The last category comes from image processing domain. It uses a combination of a computer screen and a camera. The camera monitors the exact movement and position of the entire body during exercise. The screen is used for interaction with the user.

The first category represents application such as Endomondo or Runkeeper. The second category Nike+, miCoach, Garmin Heart Belt or Fora Active tonometer. The typical representative of third category is Microsoft Kinect.

While Microsoft Kinect is designed for the indoor use all other devices are designed mostly for the outdoor use.

\section{\uppercase{Manuscript Preparation}}

\noindent We strongly encourage authors to use this document for the
preparation of the camera-ready. Please follow the instructions
% closely in order to make the volume look as uniform as possible
\cite{Moore99}.

Please remember that all the papers must be in English and without
orthographic errors.

Do not add any text to the headers (do not set running heads) and
footers, not even page numbers, because text will be added
electronically.

For a best viewing experience the used font must be Times New
Roman, except on special occasions, such as program code
\ref{subsubsec:program_code}.


\subsection{Manuscript Setup}

\noindent The template is composed by a set of 7 files, in the
following 2 groups:\\
\noindent {\bf Group 1.} To format your paper you will need to copy
into your working directory, but NOT edit, the following 4 files:
\begin{verbatim}
  - apalike.bst
  - apalike.sty
  - article.cls
  - scitepress.sty
\end{verbatim}

\noindent {\bf Group 2.} Additionally, you may wish to copy and edit
the following 3 example files:
\begin{verbatim}
  - example.bib
  - example.tex
  - scitepress.eps
\end{verbatim}


\subsection{Page Setup}

The paper size must be set to A4 (210x297 mm). The document
margins must be the following:

\begin{itemize}
    \item Top: 3,3 cm;
    \item Bottom: 4,2 cm;
    \item Left: 2,6 cm;
    \item Right: 2,6 cm.
\end{itemize}

It is advisable to keep all the given values because any text or
material outside the aforementioned margins will not be printed.

\subsection{First Section}

This section must be in one column.

\vfill
\subsubsection{Title and Subtitle}

Use the command \textit{$\backslash$title} and follow the given structure in "example.tex". The title and subtitle must be with initial letters
capitalized (titlecased). If no subtitle is required, please remove the corresponding \textit{$\backslash$subtitle} command. In the title or subtitle, words like "is", "or", "then", etc. should not be capitalized unless they are the first word of the subtitle. No formulas or special characters of any form or language are allowed in the title or subtitle.

\subsubsection{Authors and Affiliations}

Use the command \textit{$\backslash$author} and follow the given structure in "example.tex".

\subsubsection{Keywords}

Use the command \textit{$\backslash$keywords} and follow the given structure in "example.tex". Each paper must have at least one keyword. If more than one is specified, please use a comma as a separator. The sentence must end with a period.

\subsubsection{Abstract}

Use the command \textit{$\backslash$abstract} and follow the given structure in "example.tex".
Each paper must have an abstract up to 200 words. The sentence
must end with a period.

\subsection{Second Section}

Files "example.tex" and "example.bib" show how to create a paper
with a corresponding list of references.

This section must be in two columns.

Each column must be 7,5-centimeter wide with a column spacing
of 0,8-centimeter.

The section text must be set to 10-point.

Section, subsection and sub-subsection first paragraph should not
have the first line indent.

To remove the paragraph indentation (only necessary for the
sections), use the command \textit{$\backslash$noindent} before the
paragraph first word.

If you use other style files (.sty) you MUST include them in the
final manuscript zip file.

\subsubsection{Section Titles}

The heading of a section title should be in all-capitals.

Example: \textit{$\backslash$section\{FIRST TITLE\}}

\vfill
\subsubsection{Subsection Titles}

The heading of a subsection title must be with initial letters
capitalized (titlecased).

Words like "is", "or", "then", etc. should not be capitalized unless
they are the first word of the subsection title.

Example: \textit{$\backslash$subsection\{First Subtitle\}}

\subsubsection{Sub-Subsection Titles}

The heading of a sub subsection title should be with initial letters
capitalized (titlecased).

Words like "is", "or", "then", etc should not be capitalized unless
they are the first word of the sub subsection title.

Example: \textit{$\backslash$subsubsection\{First Subsubtitle\}}

\subsubsection{Tables}

Tables must appear inside the designated margins or they may span
the two columns.

Tables in two columns must be positioned at the top or bottom of the
page within the given margins. To span a table in two columns please add an asterisk (*) to the table \textit{begin} and \textit{end} command.

Example: \textit{$\backslash$begin\{table*\}}

\hspace*{1.5cm}\textit{$\backslash$end\{table*\}}\\

Tables should be centered and should always have a caption
positioned above it. The font size to use is 9-point. No bold or
italic font style should be used.

The final sentence of a caption should end with a period.

\begin{table}[h]
\caption{This caption has one line so it is
centered.}\label{tab:example1} \centering
\begin{tabular}{|c|c|}
  \hline
  Example column 1 & Example column 2 \\
  \hline
  Example text 1 & Example text 2 \\
  \hline
\end{tabular}
\end{table}

\begin{table}[h]
\caption{This caption has more than one line so it has to be
justified.}\label{tab:example2} \centering
\begin{tabular}{|c|c|}
  \hline
  Example column 1 & Example column 2 \\
  \hline
  Example text 1 & Example text 2 \\
  \hline
\end{tabular}
\end{table}

Please note that the word "Table" is spelled out.


\subsubsection{Figures}

Please produce your figures electronically, and integrate them into
your document and zip file.

Check that in line drawings, lines are not interrupted and have a
constant width. Grids and details within the figures must be clearly
readable and may not be written one on top of the other.

Figure resolution should be at least 300 dpi.

Figures must appear inside the designated margins or they may span
the two columns.

Figures in two columns must be positioned at the top or bottom of
the page within the given margins. To span a figure in two columns please add an asterisk (*) to the figure \textit{begin} and \textit{end} command.

Example: \textit{$\backslash$begin\{figure*\}}

\hspace*{1.5cm}\textit{$\backslash$end\{figure*\}}

Figures should be centered and should always have a caption
positioned under it. The font size to use is 9-point. No bold or
italic font style should be used.

\begin{figure}[!h]
  %\vspace{-0.2cm}
  \centering
   {\epsfig{file = SCITEPRESS.eps, width = 5.5cm}}
  \caption{This caption has one line so it is centered.}
  \label{fig:example1}
 \end{figure}

\begin{figure}[!h]
  \vspace{-0.2cm}
  \centering
   {\epsfig{file = SCITEPRESS.eps, width = 5.5cm}}
  \caption{This caption has more than one line so it has to be justified.}
  \label{fig:example2}
  \vspace{-0.1cm}
\end{figure}

The final sentence of a caption should end with a period.



Please note that the word "Figure" is spelled out.

\subsubsection{Equations}

Equations should be placed on a separate line, numbered and
centered.\\The numbers accorded to equations should appear in
consecutive order inside each section or within the contribution,
with the number enclosed in brackets and justified to the right,
starting with the number 1.

Example:

\begin{equation}\label{eq1}
    a=b+c
\end{equation}

\subsubsection{Program Code}\label{subsubsec:program_code}

Program listing or program commands in text should be set in
typewriter form such as Courier New.

Example of a Computer Program in Pascal:

\begin{small}
\begin{verbatim}
 Begin
     Writeln('Hello World!!');
 End.
\end{verbatim}
\end{small}


The text must be aligned to the left and in 9-point type.

\vfill
\subsubsection{Reference Text and Citations}

References and citations should follow the Harvard (Author, date)
System Convention (see the References section in the compiled
manuscript). As example you may consider the citation
\cite{Smith98}. Besides that, all references should be cited in the
text. No numbers with or without brackets should be used to list the
references.

References should be set to 9-point. Citations should be 10-point
font size.

You may check the structure of "example.bib" before constructing the
references.

For more instructions about the references and citations usage
please see the appropriate link at the conference website.

\section{\uppercase{Copyright Form}}

\noindent For the mutual benefit and protection of Authors and
Publishers, it is necessary that Authors provide formal written
Consent to Publish and Transfer of Copyright before publication of
the Book. The signed Consent ensures that the publisher has the
Author's authorization to publish the Contribution.

The copyright form is located on the authors' reserved area.

The form should be completed and signed by one author on
behalf of all the other authors.

\section{\uppercase{Conclusions}}
\label{sec:conclusion}

\noindent Please note that ONLY the files required to compile your paper should be submitted. Previous versions or examples MUST be removed from the compilation directory before submission.

We hope you find the information in this template useful in the preparation of your submission.

\section*{\uppercase{Acknowledgements}}

\noindent If any, should be placed before the references section
without numbering. To do so please use the following command:
\textit{$\backslash$section*\{ACKNOWLEDGEMENTS\}}


\vfill
\bibliographystyle{apalike}
{\small
\bibliography{citations-healthinf-2016,frontiers,bibliography}


\section*{\uppercase{Appendix}}

\noindent If any, the appendix should appear directly after the
references without numbering, and not on a new page. To do so please use the following command:
\textit{$\backslash$section*\{APPENDIX\}}

\vfill
\end{document}

